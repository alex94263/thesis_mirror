\chapter{Conclusion}\label{chap:conclusion}
In this thesis we study the relationship between selfish mining and networking effects. We evaluate the effectiveness of adversarial mining strategies in two system parameter setups validated against Bitcoin. We come to the conclusion that selfish mining is highly dependent on the network. In a setup were selfish mining is executed without network advantage it almost always outperformed by honest mining. However, if relative hashrate and network advantage are high enough selfish mining is beneficial. We find that for a relative hashrate of $45\% $ and a bandwidth 10 to 25-times faster than the average peer, selfish mining can outperform honest mining. We also observe that honest mining is dependent on relative hashrate and the network. In fact, honest mining can produce significant revenue gain with high hashrate. This gain is a result of imperfect blockchain growth, which is due to block propagation characteristics. Thus, it is reasonable to state that in most cases it is more beneficial to execute honest mining than selfish mining, unless a significant network advantage can be obtained.\\
It is up to future work to explore different adversarial mining strategies. With the astonishing performance of honest mining it might be beneficial to execute a protocol similar to honest mining, but without the obligation to propagate all blocks. If the peer has a high network influence it might be beneficial for him to exclude blocks from advertisement.\\
Overall we observe revenue gain on peers with a high relative hashrate. This implies that it is beneficial to obtain a high relative hashrate. Thus, the formation of mining pools becomes even more rewarding. However, this can lead to a reduced decentralization, if the most economically rewarding strategy is to form big mining pools. We do believe that this is not in the best interest for blockchain systems such as Bitcoin. Therefore we must explore different strategies to achieve a better growth of the blockchain.  We assume that a growth of close to 1 would reduce the revenue gain of big mining pools. The result would be a more stable decentralized cryptocurrency.