\chapter*{Zusammenfassung}
Bitcoin stellt die erfolgreichste Kryptowährung dar. Es setzt elektronisches Geld dezentral um. Dies wird ermöglicht durch Blockchain Technologie. Bitcoin stellt eine vielversprechende Alternative zu bestehenden elektronischen Banksystemen dar. Jedoch ist Bitcoin angreifbar. Eine zentrale Rolle in Bitcoin spielt Mining. Alternativ zum etablierten Honest Mining existieren bösartige Mining Verfahren, zum Beispiel Selfish Mining Diese ermöglichen es dem Angreifer einen Vorteil zu erwirtschaften, in dem er sogenannte Forks in der Blockchain forciert. Dies führt zu einer geringeren Growth der Blockchain und zu mehr verschwendeten Rechenresourcen. Diese Arbeit analysiert den Zusammenhang zwischen Selfish Mining und Netzwerkeffekten. Dazu nutzt sie ein neues analytisches Blockchain Modell, das sogenannte Selfish Rumor Model. Dieses Modell wird in einem diskreten Ereignis Simulator implementiert und gegenüber Blockverteilungscharakteristiken von Bitcoin validiert. Dies erzeugt zwei Parametersetups, welche genutzt werden um in einem Bitcoin ähnlichen Modell den Zusammenhang zu studieren. Es zeigt sich, dass Revenue nicht nur abhängig von relativer Rechenkraft ist, sondern auch vom Verhalten des Netzwerks. Selfish Mining ist stark davon abhängig wieviel Netzwerkvorteil der angreifende Peer besitzt. Aber auch Honest Mining kann einen erhöhten Revenue produzieren. Im Allgemeinen ist Selfish Mining in den meisten Fällen nicht profitabel. Es zeigt sich jedoch, dass das Studium divergierender Mining Strategien bisher nicht fokussierte Abhängigkeiten und Schwächen des Bitcoin protocols offenbart.

\chapter*{Abstract}
Bitcoin is the most prominent example for cryptocurrencies. It establishes a decentralized ledger utilizing blockchain technology offering a promising alternative to existent electronic cash systems. However, Bitcoin mining is vulnerable. Adversarial mining strategies such as selfish mining can be executed in order to gain an advantage. They result in a tilted incentive balance by forcing forks of the blockchain. Thus, they lower the growth of the blockchain and lead to wasted computational resources. In this thesis we analyze the relationship between selfish mining and networking effects. We study the relationship by implementing of a new analytic blockchain model, the Selfish Rumor Model. The simulator of the Selfish Rumor Model is validated against Bitcoins block propagation and establishes two parameter setups. Both parameter setups are used to study the impact of networking effects on selfish mining. We find that obtained revenue is not only linked to computational resources, but also linked to network behavior. Selfish mining is strongly influenced by the network advantage a peer possesses. However, honest mining can also produce an increased revenue.
Overall we come to the conclusion that selfish mining is not beneficial in most cases. However, exploring adversarial mining strategies and networking effects is important, since this exploration reveals certain dependencies and weaknesses of the Bitcoin protocol.
